\documentclass[letterpaper,11pt]{article}

\usepackage{listings}
\usepackage{color}

\definecolor{dkgreen}{rgb}{0,0.6,0}
\definecolor{gray}{rgb}{0.5,0.5,0.5}
\definecolor{mauve}{rgb}{0.58,0,0.82}

\lstset{frame=tb,
  language=Python,
  aboveskip=3mm,
  belowskip=3mm,
  showstringspaces=false,
  columns=flexible,
  basicstyle={\small\ttfamily},
  numbers=none,
  numberstyle=\tiny\color{gray},
  keywordstyle=\color{blue},
  commentstyle=\color{dkgreen},
  stringstyle=\color{mauve},
  breaklines=true,
  breakatwhitespace=true,
  tabsize=3
}

\usepackage{setspace}
\usepackage{bm}    %for textbf
\usepackage{amsmath}
\usepackage{amsfonts}   %for mathbb
\allowdisplaybreaks[4]  %from {amsmath}
\newcommand{\independent}{\rotatebox[origin=c]{90}{$\models$}}  %from {graphicx}
\usepackage{geometry}
\geometry{letterpaper, scale=0.8}  %from {geometry}
\author{Yuan Yin}
\title{EECS 545 Homework 2}
\begin{document}\large
\maketitle

\begin{lstlisting}
import numpy as np

# Process the data
z = np.genfromtxt('spambase.data', dtype = float, delimiter = ',')
np.random.seed(0) # Seed the random number generator
rp = np.random.permutation(z.shape[0]) # random permutation of indices
z = z[rp,:] # shuffle the rows of z
x = z[:,:-1]
y = z[:,-1]
x_train = x[0:2000,:]
y_train = y[0:2000]
x_test = x[2000:x.shape[0],:]
y_test = y[2000:y.shape[0]]

# Quantize variables with option 1 where values equal to the median to 2
mid = np.median(x,axis=0)
row_num = x.shape[0]
col_num = x.shape[1]
x1 = x
for i in range(row_num):
    for j in range(col_num):
        if x1[i,j] >= mid[j]:
            x1[i,j] = 2
        else:
            x1[i,j] = 1

# Quantize variables with option 2 where values equal to the median to 1
mid = np.median(x,axis=0)
row_num = x.shape[0]
col_num = x.shape[1]
x2 = x
for i in range(row_num):
    for j in range(col_num):
        if x2[i,j] >= mid[j]:
            x2[i,j] = 2
        else:
            x2[i,j] = 1
\end{lstlisting}

\end{document}