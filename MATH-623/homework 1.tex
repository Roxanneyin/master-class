\documentclass[letterpaper,11pt]{article}

\usepackage{listings}
\usepackage{color}

\definecolor{dkgreen}{rgb}{0,0.6,0}
\definecolor{gray}{rgb}{0.5,0.5,0.5}
\definecolor{mauve}{rgb}{0.58,0,0.82}

\lstset{frame=tb,
  language=Python,
  aboveskip=3mm,
  belowskip=3mm,
  showstringspaces=false,
  columns=flexible,
  basicstyle={\small\ttfamily},
  numbers=none,
  numberstyle=\tiny\color{gray},
  keywordstyle=\color{blue},
  commentstyle=\color{dkgreen},
  stringstyle=\color{mauve},
  breaklines=true,
  breakatwhitespace=true,
  tabsize=3
}

\usepackage{setspace}
\usepackage{bm}    %for textbf
\usepackage{amsmath}
\usepackage{amsfonts}   %for mathbb
\allowdisplaybreaks[4]  %from {amsmath}
\newcommand{\independent}{\rotatebox[origin=c]{90}{$\models$}}  %from {graphicx}
\usepackage{geometry}
\geometry{letterpaper, scale=0.8}  %from {geometry}
\author{Yuan Yin}
\title{MATH 623 Homework 1}
\begin{document}\large
\maketitle
\begin{spacing}{1.2}  %from {setspace}
\section*{Problem 1}
\subsection*{2)}
The main code and the function used in it:
\begin{lstlisting}
r = 0.05; N = 100; u = 2;
sigma_d = 0.5; sigma_u = 0.8;
K1 = 0.2; K2 = 20;
dt = 1/N; d = 1/u;
q_d = []; q_u = []; E_Q_range = []; E_Q = [];
for i = 1:1000
    q_u = [q_u, 1/1000*i];
    q_d = [q_d, (q_u(i)-r*dt/(u-1))*u];
    E_Q_range = [E_Q_range, q_u(i)*(u-(1+r*dt))^2 + q_d(i)*(1/u-(1+r*dt))^2 + (1-q_u(i)-q_d(i))*(1-(1+r*dt))^2];
    if  (q_u(i) <= ((u-1+u*r*dt)/(u^2-1))) & (r*dt/(u-1))
        if (E_Q_range(i) <= (sigma_u^2)) & (E_Q_range(i) >= (sigma_d^2))
            total = 0; times = 0;
            for j = 0:N % times for up
                for k = 0:(N-j) % times for hold
                    times = factorial(N)/(factorial(j)*factorial(N-j-k)*factorial(k));
                    S_N = (u^j)*(d^(N-j-k));
                    f_S_N = maxfunc(S_N,K1,K2);
                    total = total + times*(q_u(i)^j)*((1-q_u(i)-q_d(i))^k)*(q_d(i)^(N-j-k))*f_S_N/((1+r*dt)^N);
                end
            end
            E_Q = [E_Q, total];
        end
    end
end
E_Q
max(E_Q)
min(E_Q)

function y = maxfunc(x,a,b)
y = max(x-a,0)-max(x-b,0);
end
\end{lstlisting}
the result is:
\begin{lstlisting}
>> hm1_question1

ans =

    0.0532


ans =

    0.0022
\end{lstlisting}
\section*{Problem 2}
\subsection*{5)}
the main code and the function used in it:
\begin{lstlisting}
N = 100; u = 1.1; d = 0.9;
k1 = 0.5; k2 = 2;
q_u = (1-d)/(u-d); q_d = 1-q_u;
P = zeros(N+1,N+1);
for i = 1:N+1
    P(N+1,i) = payoff(u^(i-1)*d^(N-i+1),k1,k2);
end
for j = N:-1:1
    for k = 1:j
        P(j,k) =max(q_u*P(j+1,k+1)+(1-q_u)*P(j+1,k), payoff(u^(k-1)*d^(j-k),k1,k2));
    end
end
P(1,1)

function f_x = payoff(x,a,b)
if (x>0) & (x<=1)
    f_x = max(x-a,0);
elseif (x>1) & (x<=1.5)
    f_x = a;
elseif (x>1.5)
    f_x = max(b-x,0);
end
end
\end{lstlisting}
the result is:
\begin{lstlisting}
>> hm1_qestion2

ans =

    0.5000
\end{lstlisting}
\end{spacing}
\end{document}